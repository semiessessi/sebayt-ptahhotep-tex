%% packages
\usepackage[chapter]{algorithm}
\usepackage{algorithmicx}
\usepackage{algpseudocode}
\usepackage[backend=biber,style=numeric]{biblatex}
\usepackage{booktabs}
\usepackage[font=small,labelfont=bf]{caption}
\usepackage{courier} %% sets font for tt family only
\usepackage{float}
%\usepackage{fontspec}
\usepackage{epigraph}
\usepackage{glossaries}
\usepackage{graphicx}
\usepackage{imakeidx}
\usepackage{listings}
\usepackage{longtable}
\usepackage{nicematrix}
\usepackage{ntheorem}
\usepackage{parskip} %% no font indent, newlines
\usepackage[cm]{sfmath}
\usepackage{siunitx}
\usepackage{tabularx}
\usepackage{thmtools}
\usepackage{titlesec}
\usepackage{verbatim}
\usepackage{wrapfig}
\usepackage{xcolor}

%% font
\usepackage{tgadventor}
\renewcommand*\familydefault{\sfdefault} %% Only if the base font of the document is to be sans serif
\usepackage[LGR,T1]{fontenc}

%\setmathrm{Arial}
%\setmathsf{Arial}
%\setmathtt{Arial}

%% notes
\newcommand{\ssn}[1]{\textsuperscript{\textnormal{#1}}}

%% epigraph
\setlength{\epigraphwidth}{2.75in}

\newcommand{\indexed}[1]{#1\index{#1}}
\newcommand{\indexit}[1]{#1\index{#1@\textit{#1}}}
\newcommand{\sname}[1]{#1\index{#1@\textit{#1}}}
\newcommand{\ndsname}[1]{\index{#1@\textit{#1}}}
\newcommand{\nname}[1]{#1\index{#1@\textbf{#1}}}
\newcommand{\ndnname}[1]{\index{#1@\textbf{#1}}}
\newcommand{\person}[2]{#1 #2\index{#2, #1@\textit{#2, #1}}}
\newcommand{\personfirstname}[2]{#1\index{#2, #1@\textit{#2, #1}}}
\newcommand{\personlastname}[2]{#2\index{#2, #1@\textit{#2, #1}}}

%% \makeglossary
%% \makeindex[columns=2, title=Index, options= -s main.ist, intoc]

%% algorithms
\algnewcommand\algorithmicinput{\textbf{Input:}}
\algnewcommand\Input{\item[\algorithmicinput]}
\algnewcommand\algorithmicinputs{\textbf{Inputs:}}
\algnewcommand\Inputs{\item[\algorithmicinputs]}
\algnewcommand\algorithmicresult{\textbf{Result:}}
\algnewcommand\Result{\item[\algorithmicresult]}
\algnewcommand\algorithmicstart{\textbf{Start:}}
\algnewcommand\Start{\item[\algorithmicstart]}
\algrenewcommand\algorithmicindent{1.0em}

%% TODO: remove the top rule without breaking title alignment
% this fixes the top rule which is a nuisance...
%\makeatletter
%\newcommand\fs@ruled@notop{\def\@fs@cfont{\bfseries}\let\@fs@capt\floatc@ruled
	%	%\def\@fs@pre{\hrule height.8pt depth0pt \kern2pt}% <----removed
	%	\def\@fs@pre{}%
	%	\def\@fs@post{\kern2pt\hrule\relax}%
	%	\def\@fs@mid{\kern2pt\hrule\kern2pt}%
	%	\let\@fs@iftopcapt\iftrue}
%\renewcommand\fst@algorithm{\fs@ruled@notop}
%\makeatother

%% listings
\definecolor{comment}{rgb}{.4,.4,.4}
\definecolor{keyword}{rgb}{0.12, 0.12, 0.42}
\lstset{
	language=C,                            	% choose the default language of the code
	tabsize=2,
	basicstyle=\footnotesize\normalfont\ttfamily,
	commentstyle=\color{comment}\textit,
	keywordstyle=\color{keyword}\textbf,
	breaklines=true,                       	% sets automatic line breaking
	breakatwhitespace=true,                	% sets if automatic breaks should only happen at whitespace
	showspaces=false,                      	% show spaces adding particular underscores
	showstringspaces=false,                	% underline spaces within strings
	showtabs=false,                         % show tabs within strings adding particular underscores
	frame=none,                             % adds a frame around the code - none, single
	numbers=left,                          	% where to put the line-numbers -none, left, right
	numberstyle=\footnotesize,             	% the size of the fonts that are used for the line-numbers
}
\lstdefinelanguage{JavaScript}{
	keywords={typeof, new, true, false, catch, function, return, null, catch, switch, var, if, in, while, do, else, case, break, const, let},
	comment=[l]{//},
	morecomment=[s]{/*}{*/},
	morestring=[b]',
	morestring=[b]"
}
\lstdefinelanguage{Data}{
	keywords={},
}
%% theorems
\theoremstyle{break}
\newtheorem{program}{Program}
\newtheorem{example}{Example}

%% main layout and title etc.
\titleformat{\chapter}[display]{\normalfont\huge\bfseries}{\chaptertitlename\ \thechapter}{20pt}{\Huge}   
\titlespacing*{\chapter}{0pt}{-50pt}{40pt}

%% better space filling column for table
\newcolumntype{Y}{>{\centering\arraybackslash}X}

%toc depth
\setcounter{tocdepth}{2}
