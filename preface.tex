The \textit{\indexed{sebayt}} of \sname{Ptahhotep}, often called the "instructions" or "maxims" of Ptahhotep, is amongst the oldest complete work of wisdom literature that survives today.

The skill of the Ancient Egyptian scribes, engineers and artists have long captivated me. Their religious devotion to producing the finest quality of work is mind-boggling to the modern observer.

When I first began to explore the reading and writing of \indexed{hieroglyphs} I quickly noticed that, despite the ease of modern printing and digital artwork, many translations of Ancient Egyptian writings are presented in plain text, and those with hieroglyphs or hieratic text included use colourless or crude representations. They also tend not to be accompanied by images, even when the originals were.

I could not help but wonder how the ancients would make use of modern technology, and felt that they would strive to create considerably more beautiful renditions than we ordinarily do. What better work to attempt this with myself than \sname{Ptahhotep}'s teachings?

As a software engineer of some experience I produced for myself a "proof-of-concept" tool to render hieroglyphic\index{hieroglyphs} text with support for coloured glyphs and automated layout. I quickly found a set of coloured glyph images for another tool made by an enthusiast, which I could recycle for my own purposes. The colours did not adhere to the ancient standards, and some glyphs were transposed or poorly rendered, and so I created a modified version of the glyph images the use of myself or others which remedied these mistakes.

Although I originally planned to do much more work, striving to create something perfect, or at least to a very high standard, I realised that I had enough tools and experience to attempt production of an illustrated interlinear translation of \sname{Ptahhotep}'s work.

My skill with the ancient language is not fully developed yet, but with the help of dictionaries and others whose skills far surpassed mine the goal seemed achievable.

The original is rendered in \indexed{hieratic} text on \indexed{papyrus} using red and black ink. This rendition uses \indexed{hieroglyphs}, which seem to me to be the modern "high quality" and print-equivalent to the hand written hieratic of the ancient scribe. This is how the ancients rendered text on monuments and funerary goods where the highest quality was desired.

So here we have my attempt at a beautiful, faithful and useful copy and translation of the \indexed{sebayt} of \sname{Ptahhotep} as found in \indexed{Papyrus Prisse}.

