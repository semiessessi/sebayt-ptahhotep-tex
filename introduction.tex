\sname{Ptahhotep} addresses his king, \sname{Djedkare Isesi} and describes his plight, which is the suffering that comes with \indexed{old age}.\\

This is relevant to why he would want to pass down his wisdom as part of his legacy, and helps serve as an introduction to the rest of the text.\\

1 - the title \textit{imi rA} is usually translated as "overseer" but is a pun around the \textit{r} glyph having the shape of the mouth and being used for terms related to words and speech, and may mean something like "commander of words" or "commander through words".\\

2 - the \textit{niwt} sign can also determine a place, a town or other settlement and the choice of the word city is to convey a modern equivalent to this title.\\

3 - the title of \textit{TAti} translated as "vizier" is a somewhat modern projection here, it could perhaps also be translated as "prime minister", or as a "second in command" to the king.\\

4 - \textit{Xprw} is closely related to the god Khepri, who symbolises the making of progress of the sun when it rises.\\

5 - \textit{wr} as a determinative seems often to be used with concepts associated with \textit{\indexed{isfet}}, as well as being a word on its own which is usually translated to mean "great" or "old".\\

6 - \textit{iw} is a particle with no direct translation in English.

\pagebreak

\begin{tabularx}{\linewidth}{Yccc}
	\includegraphics[scale=0.5]{word-images/1-1-1-imy-ra-niwt} &
	\includegraphics[scale=0.5]{word-images/1-1-2-tAti} &
	\includegraphics[scale=0.5]{word-images/1-1-3-ptH-Htp} &
	\includegraphics[scale=0.5]{word-images/1-1-4-Ddf} \\
	\hline \\ 
	\textbf{m} \textbf{r} \textbf{niwt} t $\cdot$ &
	\textbf{TA} \textbf{t} $\cdot$ &
	\textbf{p} \textbf{t} \textbf{H} \textbf{Htp} t p A50 &
	\textbf{D} \textbf{d} \textbf{f}\\
	\hline \\
	\textit{imi rA\ssn{1} niwt\ssn{2}} & \textit{TAti\ssn{3}} & \textit{ptHHtp} & \textit{Dd.f} \\  
	\hline \\
	Overseer\ssn{1} of the city\ssn{2} & vizier\ssn{3} & \sname{Ptahhotep} & (he) says:   
\end{tabularx}

\vspace{7.5mm}

\begin{tabularx}{\linewidth}{cYcY}
	\includegraphics[scale=0.5]{word-images/1-2-1-ity} &
	\includegraphics[scale=0.5]{word-images/1-2-2-nbi} &
	\includegraphics[scale=0.5]{word-images/1-2-3-tni} &
	\includegraphics[scale=0.5]{word-images/1-2-4-xprw} \\
	\hline \\ 
	\textbf{it} \textbf{it} &
	\textbf{nb} G7 \textbf{A1} &
	\textbf{t} \textbf{n} \textbf{i} A19 &
	\textbf{Xpr} r \\
	\hline \\
	\textit{ity} & \textit{nb.i} & \textit{tni} & \textit{Xprw\ssn{4}} \\  
	\hline \\
	sovereign, & my lord, & infirmity & develops\ssn{4}
\end{tabularx}

\vspace{7.5mm}

\begin{tabularx}{\linewidth}{ccY}
	\includegraphics[scale=0.5]{word-images/1-3-1-iAw} &
	\includegraphics[scale=0.5]{word-images/1-3-2-hAw} &
	\includegraphics[scale=0.5]{word-images/1-3-3-wgg} \\
	\hline \\ 
	\textbf{i} \textbf{A} \textbf{w} A19 &
	\textbf{h} \textbf{A} \textbf{w} D54 &
	\textbf{w} \textbf{g} \textbf{g} wr\ssn{5} D54 \\
	\hline \\ 
	\textit{iAw} & \textit{hAw} & \textit{wgg} \\
	\hline \\ 
	\indexed{old age} & befalls, & feebleness
\end{tabularx}

\vspace{7.5mm}

\begin{tabularx}{\linewidth}{cYcY}
	\includegraphics[scale=0.5]{word-images/1-4-1-iww} &
	\includegraphics[scale=0.5]{word-images/1-4-2-iHw} &
	\includegraphics[scale=0.5]{word-images/1-4-3-Hr} &
	\includegraphics[scale=0.5]{word-images/1-4-4-mAw} \\
	\hline \\ 
	\textbf{w} \textbf{i} &
	\textbf{i} \textbf{H} \textbf{w} wr\ssn{5} &
	\textbf{Hr} $\cdot$ &
	\textbf{mA} A \textbf{w} Y1A \\
	\hline \\ 
	\textit{iw\ssn{6}.w} & \textit{iHw} & \textit{hr} & \textit{mAw} \\
	\hline \\ 
	comes, & weakness & is & renewed
\end{tabularx}
