\vspace*{\fill}

\sname{Ptahhotep} addresses his king, \sname{Djedkare Isesi} and describes his plight, which is the suffering that comes with \indexed{old age}.\\

This is relevant to why he would want to pass down his wisdom as part of his legacy, and helps serve as an introduction to the rest of the text.\\

\vspace*{\fill}

1 - the title \textit{imi rA} is usually translated as "overseer" but is a pun around the \textit{r} glyph having the shape of the mouth and being used for terms related to words and speech, and may mean something like "commander of words" or "commander through words".\\

2 - the \textit{niwt} sign can also determine a place, a town or other settlement and the choice of the word city is to convey a modern equivalent to this title.\\

3 - the title of \textit{TAti} translated as "vizier" is a somewhat modern projection here, it could perhaps also be translated as "prime minister", or as a "second in command" to the king.\\

4 - \textit{Xprw} is closely related to the god Khepri, who symbolises the making of progress of the sun when it rises.\\

5 - \textit{wr} as a determinative seems often to be used with concepts associated with \textit{\indexed{isfet}}, as well as being a word on its own which is usually translated to mean "great" or "old". e.g. \textit{Hrw wr} for "Horus the Elder" and \textit{mHt wrt} for "the Great Flood" \\

6 - \textit{iw} is a particle with no direct translation in English.

\vspace*{\fill}

\pagebreak

\vspace*{\fill}

\begin{tabularx}{\linewidth}{cYcc}
	\includegraphics[scale=0.5]{word-images/1-1-1-imy-ra-niwt} &
	\includegraphics[scale=0.5]{word-images/1-1-2-tAti} &
	\includegraphics[scale=0.5]{word-images/1-1-3-ptH-Htp} &
	\includegraphics[scale=0.5]{word-images/1-1-4-Ddf} \\
	\hline \\ 
	\textbf{m} \textbf{r} \textbf{niwt} t $\cdot$ &
	\textbf{TA} \textbf{t} $\cdot$ &
	\textbf{p} \textbf{t} \textbf{H} \textbf{Htp} t p A50 &
	\textbf{D} \textbf{d} \textbf{f}\\
	\hline \\
	\textit{imi rA\ssn{1} niwt\ssn{2}} & \textit{TAti\ssn{3}} & \textit{ptHHtp} & \textit{Dd.f} \\  
	\hline \\
	Overseer\ssn{1} of the city\ssn{2} & vizier\ssn{3} & \sname{Ptahhotep} & (he) says:   
\end{tabularx}

\vspace{7.5mm}

\begin{tabularx}{\linewidth}{cYcY}
	\includegraphics[scale=0.5]{word-images/1-2-1-ity} &
	\includegraphics[scale=0.5]{word-images/1-2-2-nbi} &
	\includegraphics[scale=0.5]{word-images/1-2-3-tni} &
	\includegraphics[scale=0.5]{word-images/1-2-4-xprw} \\
	\hline \\ 
	\textbf{it} \textbf{it} &
	\textbf{nb} G7 \textbf{A1} &
	\textbf{t} \textbf{n} \textbf{i} A19 &
	\textbf{xpr} r \\
	\hline \\
	\textit{ity} & \textit{nb.i} & \textit{tni} & \textit{xprw\ssn{4}} \\  
	\hline \\
	sovereign, & my lord, & infirmity & develops\ssn{4}
\end{tabularx}

\vspace{7.5mm}

\begin{tabularx}{\linewidth}{ccY}
	\includegraphics[scale=0.5]{word-images/1-3-1-iAw} &
	\includegraphics[scale=0.5]{word-images/1-3-2-hAw} &
	\includegraphics[scale=0.5]{word-images/1-3-3-wgg} \\
	\hline \\ 
	\textbf{i} \textbf{A} \textbf{w} A19 &
	\textbf{h} \textbf{A} \textbf{w} D54 &
	\textbf{w} \textbf{g} \textbf{g} wr\ssn{5} D54 \\
	\hline \\ 
	\textit{iAw} & \textit{hAw} & \textit{wgg} \\
	\hline \\ 
	\indexed{old age} & befalls, & feebleness
\end{tabularx}

\vspace{7.5mm}

\begin{tabularx}{\linewidth}{cYcY}
	\includegraphics[scale=0.5]{word-images/1-4-1-iww} &
	\includegraphics[scale=0.5]{word-images/1-4-2-iHw} &
	\includegraphics[scale=0.5]{word-images/1-4-3-Hr} &
	\includegraphics[scale=0.5]{word-images/1-4-4-mAw} \\
	\hline \\ 
	\textbf{w} \textbf{i} &
	\textbf{i} \textbf{H} \textbf{w} wr\ssn{5} &
	\textbf{Hr} $\cdot$ &
	\textbf{mA} A \textbf{w} Y1A \\
	\hline \\ 
	\textit{iw\ssn{6}.w} & \textit{iHw} & \textit{Hr} & \textit{mAw} \\
	\hline \\ 
	comes, & weakness & is & renewed
\end{tabularx}

\vspace*{\fill}

\pagebreak

\vspace*{\fill}

\sname{Ptahhotep} continues to lament his \indexed{old age}, describing and analogising his difficulties.

\vspace*{\fill}

7 - literally "all suns", \textit{rA} meaning "sun" and \textit{nb} used as a suffix meaning "all". \textit{rA} is identical in spelling and form to the name of the sun god \nname{Ra}.

8 - the \textit{-ti} and \textit{-wi} endings signify \indexed{dual}s, which are here translated by saying "both", whereas in English one might say "the eyes" or "the ears". In the original hieratic the singular signs are duplicated to convey the dual, and so the transcription here does the same.

9 - \textit{\indexed{ib}} is directly translated as heart, but the ancient Egyptians considered the heart to be the seat of intelligence and decision making, much as we today think about the brain.

10 - the word \textit{r} seems to be used for both the mouth and for speech.

11 - speech is only implied, this construct seems to mean something more like the colloquial or artifical english construct "wording" or "to do words".

\vspace*{\fill}

\pagebreak

\vspace*{\fill}

\begin{tabularx}{\linewidth}{cYc}
	\includegraphics[scale=0.5]{word-images/1-5-1-sDrnf} &
	\includegraphics[scale=0.5]{word-images/1-5-2-Xrdw} &
	\includegraphics[scale=0.5]{word-images/1-5-3-rA-nb} \\
	\hline \\ 
	\textbf{s} \textbf{Dr} r A55 \textbf{n} \textbf{f} &
	\textbf{X} \textbf{d} \textbf{r} wr &
	\textbf{r} \textbf{A} rA \textbf{nb} \\
	\hline \\ 
	\textit{sDr.n f} & \textit{Xrd.w} & \textit{rA nb} \\
	\hline \\ 
	one sleeps & like a child & every day\ssn{7}
\end{tabularx}

\vspace{7.5mm}

\begin{tabularx}{\linewidth}{cYcY}
	\includegraphics[scale=0.5]{word-images/1-6-1-irti} &
	\includegraphics[scale=0.5]{word-images/1-6-2-nDsw} &
	\includegraphics[scale=0.5]{word-images/1-6-3-Anxwi} &
	\includegraphics[scale=0.5]{word-images/1-6-4-imrw} \\
	\hline \\ 
	\textbf{ir} \textbf{ir} &
	\textbf{n} \textbf{D} \textbf{s} \textbf{W} wr &
	\textbf{Anx} \textbf{Anx} sDm sDm &
	\textbf{i} \textbf{mr} \textbf{w} sDm \\
	\hline \\ 
	\textit{irti} & \textit{nDs.w} & \textit{Anxwi} & \textit{imr.w} \\
	\hline \\ 
	both\ssn{8} eyes & blind, & both\ssn{8} ears & deaf
\end{tabularx}

\vspace{7.5mm}

\begin{tabularx}{\linewidth}{ccYYc}
	\includegraphics[scale=0.5]{word-images/1-7-1-phtiw} &
	\includegraphics[scale=0.5]{word-images/1-4-3-Hr} &
	\includegraphics[scale=0.5]{word-images/1-7-3-aqn} &
	\includegraphics[scale=0.5]{word-images/1-7-4-wrd} &
	\includegraphics[scale=0.5]{word-images/1-7-5-ibi} \\
	\hline \\ 
	\textbf{p} \textbf{H} pHt \textbf{t} \textbf{w} A3 &
	\textbf{Hr} $\cdot$ &
	\textbf{a} \textbf{q} wr \textbf{ni} &
	\textbf{wr} r \textbf{d} A2 &
	\textbf{ib} $\cdot$ \textbf{A1} \\
	\hline \\ 
	\textit{pHtiw} & \textit{Hr} & \textit{Aq.n} & \textit{wrd} & \textit{\indexed{ib}.i} \\
	\hline \\ 
	strength & is & waning, & tired & my heart\ssn{9}
\end{tabularx}

\vspace{7.5mm}

\begin{tabularx}{\linewidth}{cccY}
	\includegraphics[scale=0.5]{word-images/1-8-1-r} &
	\includegraphics[scale=0.5]{word-images/1-8-2-grw} &
	\includegraphics[scale=0.5]{word-images/1-8-3-ni} &
	\includegraphics[scale=0.5]{word-images/1-8-4-mdwnf} \\
	\hline \\ 
	\textbf{r} $\cdot$ &
	\textbf{g} \textbf{r} A1 &
	\textbf{ni} &
	\textbf{md} d \textbf{w} A1 \textbf{n} \textbf{f} \\
	\hline \\ 
	\textit{r} & \textit{gr.w} & \textit{ni} & \textit{mdw.n f} \\
	\hline \\ 
	mouth\ssn{10} & is silent, & not & speaking\ssn{11} words
\end{tabularx}

\vspace*{\fill}

\pagebreak

\vspace*{\fill}

\sname{Ptahhotep}'s lament continues.

\vspace*{\fill}

12 - this reading is uncertain.

13 - this is not an obvious translation, and the pieces referred to are not explicitly body parts. This could also be translated as places or things, although the usual word for things is \textit{xt}.

\vspace*{\fill}

\pagebreak

\vspace*{\fill}

\begin{tabularx}{\linewidth}{cYcYc}
	\includegraphics[scale=0.5]{word-images/1-9-1-ib} &
	\includegraphics[scale=0.5]{word-images/1-9-2-tmw} &
	\hspace{4mm}\includegraphics[scale=0.5]{word-images/1-8-3-ni} &
	\includegraphics[scale=0.5]{word-images/1-9-4-sXAnf} &
	\includegraphics[scale=0.5]{word-images/1-9-5-sf} \\
	\hline \\ 
	\textbf{ib} $\cdot$ &
	\textbf{tm} m \textbf{W} wr &
	\hspace{4mm}\textbf{ni} &
	\textbf{s} \textbf{X} \textbf{A} \textbf{n} \textbf{f} &
	\textbf{sf} rA \\
	\hline \\ 
	\textit{ib} & \textit{tm.w} & \hspace{4mm}\textit{ni} & \textit{sXA.n f} & \textit{sf} \\
	\hline \\ 
	heart & fails, & \hspace{3mm}not & remembering & yesterday
\end{tabularx}

\vspace{7.5mm}

\begin{tabularx}{\linewidth}{cYc}
	\includegraphics[scale=0.5]{word-images/1-A-1-qs} &
	\includegraphics[scale=0.5]{word-images/1-A-2-mnnf} &
	\includegraphics[scale=0.5]{word-images/1-A-3-Aww} \\
	\hline \\ 
	\textbf{q} \textbf{s} T19 &
	\textbf{mn} n \textbf{n} wr \textbf{f} \textbf{n} &
	\textbf{Aw} \textbf{w} rA Y1A \\
	\hline \\ 
	\textit{qs} & \textit{mn.n f n} & \textit{Aww} \\
	\hline \\ 
	bones & hurt from & high age\ssn{12}
\end{tabularx}

\vspace{7.5mm}

\begin{tabularx}{\linewidth}{cYccYc}
	\includegraphics[scale=0.5]{word-images/1-B-1-bw} &
	\includegraphics[scale=0.5]{word-images/1-B-2-nfr} &
	\includegraphics[scale=0.5]{word-images/1-B-3-xpr} &
	\includegraphics[scale=0.5]{word-images/1-B-4-m} &
	\includegraphics[scale=0.5]{word-images/1-B-1-bw} &
	\includegraphics[scale=0.5]{word-images/1-B-6-bin} \\
	\hline \\ 
	\textbf{b} \textbf{W} &
	\textbf{nfr} &
	\textbf{xpr} &
	\textbf{n} &
	\textbf{b} \textbf{W} &
	\textbf{b} \textbf{i} \textbf{n} wr \\
	\hline \\ 
	\textit{bw} & \textit{nfr} & \textit{xpr} & \textit{m} & \textit{bw} & \textit{bin} \\
	\hline \\ 
	pieces\ssn{13} & beautiful & develop & to & pieces\ssn{13} & evil
\end{tabularx}

\vspace*{\fill}
